\documentclass[11pt]{article}
\usepackage[left=2cm,right=2cm,bottom=2cm, top=1cm]{geometry}
\usepackage{amsmath,amssymb,indentfirst}
%\setlength{\parskip}{0.4em}
\usepackage{titlesec}
\titleformat*{\section}{\large\it\notag}
\titlespacing*{\section}{0pt}{0.5em}{0.5em}


\begin{document}
\begin{center}
\it
\large
Research proposal by
Fedor K. Popov

\end{center}

My research interests encompass a variety of cutting-edge problems in Theoretical Physics, ranging from Many-Body Scar states in fermionic lattice models to applications of Quantum Field Theory (QFT) to cosmology. QFT is widely regarded as the universal language of theoretical physics, and it is of great importance in modern science. 
I am very interested in the applications of QFT to statistical physics, high-energy physics, and quantum gravity.
My current research is based on different approaches that use novel ideas which result in a better understanding of certain limits of complicated problems.

Specifically, there are three particular areas of QFT that I would like to focus on. Namely, these are: quantum field theories in the large $N$ limit, fermionic matrix and vector models and quantum field theories in  curved spacetimes. As a part of Simons Junior Fellowship I would relish the opportunity to further develop my research on these topics. The independence
offered by the Fellowship would allow me to explore these topics and increase my knowledge from a wide circle of great scientists.
%and deepen my understanding of the various fields.

\section*{Quantum Field Theories in the large $N$ limit}
One of the most promising and long-standing approaches %to solve problems 
in quantum field theory is to study systems with a large number of degrees of freedom (often referred to as the ''large $N$ limit''). 
Under the guidance of Professor Igor Klebanov I have been studying tensor models where a new, unique large $N$ limit was found.
%I would like to continue my research on this topic in Harvard%, 
Delving deeper into these models would be fruitful and beneficial, as they have already produced a significant number of new, prominent results that continue to provide a deeper understanding of the AdS/CFT correspondence. 

My research in this area uses standard techniques of QFT such as the renormalization group flows and Wilson-Fisher $\epsilon$-expansion to investigate the properties of the tensor models.
We managed to show that in the large $N$ limit, the exact solution for arbitrary dimensions is in agreement with $\epsilon$-expansion, which gives an answer for any $N$, but only in the vicinity of the critical dimension. % dimensions close-to-critical.
This approach could help to better understand the subleading orders in the $1/N$-expansion of tensor models.

%Working with Professors Giombi, Klebanov and others we managed to use such an approach to investigate the "prismatic" models - models that could be thought of as generalizations of the scalar tensor model, but with a potential that is bounded below. Due to the positivity of the potential, we managed to avoid the %ubiquitous 
%problem of complex dimensions arising in scalar tensor models. As a continuation of this research, I have written a paper, where a supersymmetric tensor model was worked out in a similar fashion. This model, due to supersymmetry, gave a consistent conformal field theory as a stable IR-fixed point that is in agreement with exact solution in the large $N$ limit.

%Currently, with Professor Klebanov and Christian Jepsen %of Stony Brook 
%we are studying the $\epsilon$-expansion of
%the tensor and matrix scalar models, where the fundamental fields transform as irreducible representations of $O(N)$ group. We have already found crucial differences in the large $N$ results, depending on the representation.  Namely, the anti-symmetric tensor models correctly reproduce the results of the usual large $N$ models.  Meanwhile, in the case of symmetric representations, we get a complex CFT, indicating an instability of the theory, thus giving a simple counter-example to the orbifold conjecture. 


There are numerous open questions and interesting avenues of exploration that could be undertaken in this field of study. These tensor models, being an exemplar of the celebrated SYK model - without disorder - could bring a deeper understanding in both high-energy and condensed matter physics.

\section*{Fermionic models and Many-Body Scar States}
Fermionic matrix, vector and tensor models are another intriguing topic in theoretical physics since they are deeply connected with models of condensed matter physics, most notably, the SYK model. With Professor Klebanov and others, we investigated the spectrum of such models through analytical means. 

The reduction of the tensor fermionic model to the matrix or vector cases yielded exact integrability. Namely, the energies and their degeneracies could be computed through representation theory. In the case of the vector model, we found that the spectrum has a Hagedorn behavior. The system has a maximal temperature, which is quite novel for one-dimensional models. All of these results provide further insight into finite $N$ SYK and tensor models. 


The deformation of such models by special operators resulted in the decoupling of the group-invariant sector from the rest of the system, revealing the connection with the recently discovered "scar" states. Because of the construction the singlet states are protected from the external impact and do not thermalize.
This property is crucial for applications to quantum computers and presents an interesting violation of the Eigenstate Thermolization Hypothesis (ETH).


I would like to continue exploring these models and unraveling their intimate connection with the AdS/CFT correspondence, high-$T_C$ superconductivity,
%condensed matter physics,
representation theory and combinatorics. %For example, a famous Cauchy identity for Schur polynomials could be derived as a consequence of the Pauli exclusion principle.
%Together with direct connection to condensed matter physics it could be very beneficial to the problems of AdS/CFT correspondence and high-$T_C$ superconductivity.


\section*{Quantum Field Theories in curved space-time}
The study of QFT in curved spacetimes is one of the important areas of the modern theoretical physics, since it intertwines the unique phenomena of the cosmological constant, the AdS/CFT correspondence, and wormhole solutions. Together with Professor Maldacena and Alexei Milekhin we investigated the creation and stabilization of magnetic wormholes. Under the guidance of Professor Polyakov, I studied the behavior of field theories and stability of various states in de-Sitter space.

%We managed to show, that the BD state even though being a thermal one, is quite peculiar since it does not posses usual properties of thermal states in flat space.
%Also together with Professor Polyakov, we studied the problem of the analytical continuation from the compact spaces to non-compact ones, that could possibly indicate the problems of the stability of de Sitter space.

%I would like to continue exploring both of these topics. % In particular, it would be interesting to study the properties of such magnetic wormholes from the point of view of chiral effects.
%The stability of de-Sitter space is in general a more fundamental problem, since it represents our Universe and could answer the outstanding questions of the cosmological problem and creation of the universe.

\end{document}
